In conclusion the analyzed system is robust and well done, even if some details need to be fixed.
Some problems that we revealed are: \\
\begin{itemize}
    \item The project's ITD document states that all addresses are in region Lombardia for simplicity.
    Therefore, all addresses that we create were in Milan.
    However, even though we have explicitly stated the province code (MI) in the address, the location database tried to match the address to Bergamo (BG).
    \item Following the previous issue, the Geolocation API failed to execute our addresses multiple times, we had to implement a retry logic for this.
    \item Also, regarding the same issue we have detected that a certain failure causes the app to crash completely.
    We had to alter the code to prevent the crash so that our tests are replicable (Line 24 in /src/controller/services/StoreManagerServices.js)
    \item The queue management is too simple since it just set the ETA to 5 minutes for each person before in the queue even if the maximum store capacity is not reached.
    This causes an unnecessary delay in the store exiting and doesn't consider the size of the store.
    \item There is no error when the manager reduces the shop's capacity below the amount of customers already present in the store.
    Even though this is not a regular case to happen, this is not enlisted as impossible in Domain Assumptions, and it is not mentioned as a requirement.
\end{itemize}
