Acceptance tests are divided in six sections, one for each functionality described in the ITD document:

\subsection{createStoreManager}

\begin{itemize}
    \item Will ask for the necessary information to register a user (generalization of Store Manager): \\
    Test related to the requirement R1 of the RASD document, \\
    Test steps:
    \begin{enumerate}
        \item Go to register store page,
        \item Check if the fields email, name, username, password, vat, address, store name and capacity exists.
    \end{enumerate}
    Test Passed \\

    \item Will register a person to the system as a user (generalization of Store Manager): \\
    Test related to the requirement R1 of the RASD document, \\
    Test steps:
    \begin{enumerate}
        \item Go to register store page,
        \item Fill register form with valid data,
        \item Click register button,
        \item Check if the url the user is redirected contains word "login™,
        \item Check if the status code for login is 200.
    \end{enumerate}
    Test Passed\\

    \item Will verify that the email provided by a person during the registration process is unique (generalization of Store Manager): \\
    Test related to the requirement R2 of the RASD document, \\
    Test steps:
    \begin{enumerate}
        \item Go to register page,
        \item Fill register form with valid data,
        \item Click register button,
        \item Clear cookies,
        \item Go to register page,
        \item Fill register form with same email,
        \item Click register button,
        \item Check if the url the user is redirected contains word "register",
        \item Check if the element "toast" exists with description "This email is already in use.".
    \end{enumerate}
    Test Passed \\

    \item Will verify that the VAT number provided by a store manager during the registration process is unique: \\
    Test related to the requirement R3 of the RASD document, \\
    Test steps:
    \begin{enumerate}
        \item Go to register page,
        \item Fill register form with valid data,
        \item Click register button,
        \item Clear cookies,
        \item Go to register page,
        \item Fill register form with same vat number,
        \item Click register button,
        \item Check if the element "toast" exists with description "This VAT number is already in use.".
    \end{enumerate}
    Test Passed \\

    \item Will create a new store after the registration of store manager: \\
    Test related to the requirement R4 of the RASD document, \\
    Test steps:
    \begin{enumerate}
        \item Go to register page,
        \item Fill register form with valid data,
        \item Click register button,
        \item Clear cookies,
        \item Check if there is a "Sign in" link,
        \item Sign in with registered data,
        \item Check if the url the user is redirected contains word "overview",
        \item Check if the element "h1" exists with description "Store overview".
    \end{enumerate}
    Test Passed \\

    \item A user (generalization of Store Manager) is able to log into the system by entering his personal credentials: \\
    Test related to the requirement R5 of the RASD document, \\
    Test steps:
    \begin{enumerate}
        \item Go to register page,
        \item Fill register form with valid data,
        \item Click register button,
        \item Clear cookies,
        \item Check if there is a "Sign in" link,
        \item Sign in with registered data,
        \item Check if the url the user is redirected contains word "overview",
    \end{enumerate}
    Test Passed \\

\end{itemize}
\subsection{createUser}

\begin{itemize}
    \item Will ask for the necessary information to register a user (generalization of Clupper): \\
    Test related to the requirement R1 of the RASD document. \\
    Test steps:
    \begin{enumerate}
        \item Click on the register button,
        \item Check that there are input field for email, password, name and surname.
    \end{enumerate}
    Test Passed \\

    \item Will register a person to the system as a user (generalization of Clupper): \\
    Test related to the requirement R1 of the RASD document. \\
    Test steps:
    \begin{enumerate}
        \item Click on the register button,
        \item Insert random strings in the fields,
        \item Click on register,
        \item Log in using the same email and password,
        \item Checks if the response from the server has an http status code of 200.
    \end{enumerate}
    Test Passed \\

    \item Will verify that the email provided by a person during the registration process is unique (generalization of Clupper):\\
    Test related to the requirement R2 of the RASD document. \\
    Test steps:
    \begin{enumerate}

        \item Click on the register button,

        \item Insert random strings in the fields,

        \item Click on register,

        \item Clear the cookies to invalidate the actual session,

        \item Return to the home page,

        \item Click on the register button,

        \item Insert the same email in the email field,

        \item Insert random strings in the other fields,

        \item Click on register,

        \item Checks that an error message is shown.
    \end{enumerate}
    Test Passed \\

    \item A user (generalization of Clupper) is able to log into the system by entering his personal credentials: \\
    Test related to the requirement R5 of the RASD document. \\
    Test steps: \\
    \begin{enumerate}

        \item Click on the register button,

        \item Insert random strings in the fields,

        \item Click on register,

        \item Clear the cookies to invalidate the actual session,

        \item Return to the home page,

        \item Click on the sign in button,

        \item Insert the same email and password,

        \item Click on the login button,
        \item Checks that the system land on the page /explore that is the main page of a logged in user.
    \end{enumerate}
    Test Passed

\end{itemize}
\subsection{joinDigitalQueue}

Before this tests a customer is registered and logged in.

\begin{itemize}
    \item A clupper is able to join a queue \\
    Test related to the requirement R7 of the RASD document. \\
    Test steps:\\

    \begin{enumerate}
        \item Save the store name and number of customers of the first store in the list,
        \item Click on the first store in the list,
        \item Check if the name of the store and the number of custumer are present in the new page.
    \end{enumerate}
    Test Passed

    \item The system is able to insert a ticket into a store queue \\
    Test related to the requirement R8 of the RASD document. \\
    Test steps: \\
    \begin{enumerate}

        \item A new manager (with related store) is created,

        \item A new customer is created and logged in,

        \item Click on the new store,

        \item Click on the button to join the queue,

        \item Click on the button to return to the main page(/explore),

        \item Check that the store has one person in line.
    \end{enumerate}
    Test Passed

    \item A clupper is able to retrieve a previously obtained ticket \\
    Test related to the requirement R16 of the RASD document. \\
    Test steps: \\
    \begin{enumerate}

        \item A new manager (with related store) is created,

        \item A new customer is created and logged in,

        \item Click on the new store,

        \item Click on the button to join the queue,

        \item Click on the button to return to the main page(/explore),

        \item Logout,

        \item Log in with the same customer,

        \item Click on the button for seeing the queue,

        \item Check that the ticket is the same as before.
    \end{enumerate}
    Test Passed

    \item The system is able to remove a ticket from a store queue \\
    Test related to the requirement R9 of the RASD document. \\
    Test steps:\\
    \begin{enumerate}
        \item A new manager (with related store) is created,

        \item A new customer is created and logged in,

        \item Click on the new store,

        \item Click on the button to join the queue,

        \item Click on the button to return to the main page(/explore),

        \item Log out,

        \item Log in with the same user,

        \item Click on the button to see the tickets,

        \item Click on the button to delete the ticket,

        \item Reclick on the button to see the tickets,

        \item Checks that an error message that says that there are no ticket is shown.
    \end{enumerate}
    Test Passed

    \item A clupper is able to join at most one queue at any time (for any store)\\
    Test related to the requirement R7 of the RASD document.\\
    Test steps:\\
    \begin{enumerate}
        \item Create 2 different managers responsible for 2 different stores
        \item Login as a new customer
        \item Add the customer to the queue for the first store
        \item Open the addition page for the second store.
        \item Verify that the button for adding the customer is disabled.
        \item Click the verify button.
        \item Verify that an error message is shown indicating that the customer can join at one queue at a time.
    \end{enumerate}
    Test Passed
\end{itemize}

\subsection{joinPhysicalQueue}

\begin{itemize}
    \item R6 - The system is able to generate a new ticket after receiving a request\\
    Test verifies that the system can generate a ticket upon the request by a manager. \\
    Test steps: \\
    \begin{enumerate}

        \item Login as a manager,

        \item Save the address,

        \item Click on the button for issuing tickets,

        \item Check that a QR code exist,

        \item Check that the address in this page is the same as before.
    \end{enumerate}
    Test Passed

    \item Check if the ticket cannot be closed before it is printed\\
    Test used to check if the system prevent the manager from closing the page with the ticket before printing it. \\
    We couldn't find a requirement for this, however this feature was implemented in the code. \\
    Test steps: \\
    \begin{enumerate}
        \item Login as a manager,

        \item Click on the button for issuing tickets,

        \item Click on the close button,

        \item Checks that an error message is shown.
    \end{enumerate}
    Test Passed \\

    \item Check if the ticket can be printed: \\
    Test used to check if the system allow the manager to print a new ticket. \\
    We couldn't find a requirement for this, however it is mentioned as a function in the RASD. \\
    Test steps:\\
    \begin{enumerate}
        \item Login as a manager,

        \item Click on the button for issuing tickets,

        \item Check if exists a button for printing,

        \item Click on the print button,

        \item Check that the property calledOnce is set to true.
    \end{enumerate}
    Test Passed \\

    \item R9 - The system is able to remove a ticket from a store queue \\
    Test verifies that a manager can delete a ticket from the store queue. \\
    Test steps: \\
    \begin{enumerate}
        \item Login as a manager,

        \item Click on the button for issuing tickets,

        \item Check if exists a button for printing,

        \item Click on the print button,
    \end{enumerate}
    Test Passed\\

    \item R31 - The system is able to retrieve the number of customers currently in a store queue \\
    Test used to check if creating a new ticket actually increment the size of the queue. \\
    Test steps: \\
    \begin{enumerate}
        \item A new manager (with related store) is created,

        \item Login as the new manager,

        \item Checks that the number of customer in the queue is 0,

        \item Click on the button for issuing tickets,

        \item Click on the print button,

        \item Click on the close button,

        \item Checks that the number of customer in the queue is incremented to 1.
    \end{enumerate}
    Test Passed

    \item R6 - The system is able to generate a new ticket after receiving a request even when the store is full \\
    Test used to check if the system allows the manager to issue more tickets than the store capacity. \\
    We wanted to check this given that the ability to create tickets and check in can easily be mixed by developers. \\
    Test steps: \\
    \begin{enumerate}
        \item Create a new store with capacity 0,
        \item Login with the manager of that store,
        \item Check that the capacity is 0,
        \item Click on the printing button,
        \item Click on the close button,
        \item Check that the number of customer in the queue is 1
    \end{enumerate}
    Test Passed\\
\end{itemize}

\subsection{storeOverview}

Before this tests a manager is registered and logged in.

\begin{itemize}

    \item The system is able to retrieve the number of customers currently in a store queue, The system is able to retrieve the number of customers currently inside a store: \\
    Test related to the requiremnts R31 and R32 of the RASD document. \\
    Test steps: \\
    \begin{enumerate}
        \item Login as a customer,

        \item Get a ticket from the store of the manager registered at the beginning,
        \item Login as manager,
        \item Check that there is 1 person in the queue and 0 in the store,
        \item Login as another customer,
        \item Get another ticket from the store of the manager registered at the beginning,
        \item Login as manager,
        \item Check that there are 2 persons in the queue and 0 in the store,
        \item Validate the first ticket,
        \item Check that there is 1 person in the queue and 1 in the store,
        \item Login as another customer,
        \item Get another ticket from the store of the manager registered at the beginning,
        \item Login as manager,
        \item Check that there are 2 persons in the queue and 1 in the store,
        \item Validate the second ticket,
        \item Check that there is 1 person in the queue and 2 in the store,
        \item Validate the exit of the first ticket,
        \item Check that there is 1 person in the queue and 1 in the store,
        \item Validate the exit of the third ticket,
        \item Check that there are 0 persons in the queue and 2 in the store.
    \end{enumerate}
    Test Passed \\

    \item A store manager is able to change the maximum store capacity at any time: \\
    Test related to the requirement R17 of the RASD document. \\
    Test steps: \\
    \begin{enumerate}
        \item Create a new store with capacity 3,
        \item Login as customer,
        \item Get a ticket from the store registered at the beginning,
        \item Login as another customer,
        \item Get another ticket from the store registered at the beginning,
        \item Login as manager,
        \item Validate the first ticket,
        \item Check that the capacity of the store is 3,
        \item Click on the update button,
        \item Write 5 as input for the capacity,
        \item Click the accept button,
        \item Chack that the capacity of the store is 5.
    \end{enumerate}
    Test Passed \\

    \item Store capacity can not be reduced below the current amount of customers inside: \\
    Test used to check if the system prevents the manager to change the maximum number of user in the store below the actual number of the customers in the store.\\
    Test steps:\\
    \begin{enumerate}
        \item Login as a customer,

        \item Get a ticket from the store of the manager regstered at the beginning,

        \item Login as a second customer,

        \item Get a ticket from the store of the manager regstered at the beginning,

        \item Login as a manager,

        \item Scan the tickets,

        \item Update the capacity of the store to 1,

        \item Check if a error message is shown.
    \end{enumerate}
    Test Failed: \\
    There is no indication in the document on what should happen if the amount of shoppers in store are larger than the new capacity.
    We expect some sort of an error, but it just continues with an illegal text like 2/1.
\end{itemize}

\subsection{ticketScan}

\begin{itemize}
    \item R19 - A store manager is able to scan a customer’s ticket at the entrance of the store before letting him in\\
    R22 - The system is able to remove a customer’s ticket from the store queue when it is scanned at the entrance\\
    R21 - The system is able to perform a validity check on a ticket scanned by a store manager and to inform him about the result\\
    Test is used to check whether scanning a customer's ticket allows access and updates the customer report accordingly. \\
    Test steps:\\
    \begin{enumerate}
        \item Create a new store by registering as a manager
        \item Create and login as a new customer
        \item Generate a ticket for the customer
        \item Login as the store manager to validate the ticket
        \item Check that a success message is displayed and the numbers reported are updated to reflect the entrance.
    \end{enumerate}
    Test Passed
    \item R23 - A customer's ticket is scanned at the exit \\
    Test is used to check whether a customer's exit from the store is reflected on the store numbers.
    Test steps: \\
    \begin{enumerate}
        \item Create a new store by registering as a manager
        \item Create and login as a new customer
        \item Generate a ticket for the customer
        \item Login as the store manager to validate the ticket twice, which corresponds to checking in and out
        \item Check that a success message is displayed and the numbers reported are updated to reflect the exit of the customer.
    \end{enumerate}
    Test Passed \\
    \item R19 - A store manager is able to scan a customer's ticket only if the number of customers currently inside does not exceed its maximum capacity \\
    R21 - The system is able to perform a validity check on a ticket scanned by a store manager and to inform him about the result \\
    Test is used to check whether the manager can not scan another customer in when the maximum store capacity is reached.
    Test steps: \\
    \begin{enumerate}
        \item Create a new store with a capacity 1 by registering as a manager
        \item Create and login as a new customer
        \item Generate a ticket for the customer
        \item Create and login as another customer
        \item Generate a ticket for that customer customer
        \item Login as the store manager to validate both of the tickets
        \item Check that an error message is displayed and the numbers reported do not exceed the set capacity.
    \end{enumerate}
    Test Passed\\
    \item R23 - The system is able to invalidate a customer's ticket \\
    R21 - The system is able to perform a validity check on a ticket scanned by a store manager and to inform him about the result \\
    Test is used to detect that the customers ticket is invalid after exiting from the store \\
    Test steps:
    \begin{enumerate}
        \item Create a new store by registering as a manager
        \item Create and login as a new customer
        \item Generate a ticket for the customer
        \item Login as the store manager to validate the ticket twice, which corresponds to checking in and out and trying an invalid ticket.
        \item Check that an error message is displayed and the numbers reported are updated to reflect the exit of the customer, but not a change for the third scan.
    \end{enumerate}
    Test Passed\\
    \item No one should be able to enter before his turn (We can't find a specific requirement for this) \\
    R21 - The system is able to perform a validity check on a ticket scanned by a store manager and to inform him about the result \\
    Test is used to verify that the ordering is assured, even though the requirements don't mention this, it is specified in the function definition and various parts of the RASD.\\
    Test steps:
    \begin{enumerate}
        \item Create a new store with a capacity 1 by registering as a manager
        \item Create and login as a new customer
        \item Generate a ticket for the customer
        \item Create and login as another customer
        \item Generate a ticket for that customer customer
        \item Login as the store manager to validate the second ticket
        \item Check that an error message is displayed indicating that the customer is not at the head of the queue.
    \end{enumerate}
    Test Passed\\
\end{itemize}
