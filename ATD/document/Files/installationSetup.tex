\subsection{Project Setup}
To install the prototype we needed to:
\begin{enumerate}
    \item install NodeJS, (download from the official website)
    \item install the dependencies, (via \textit{npm i})
    \item run the application using a custom script provided by the developers (\textit{npm run start})
\end{enumerate}
The installation guide is easy to follow and complete, but there are some details that can be specified better.
For example the specific minimum version of NodeJS that needs to be installed could have been specified, as two of our group members had problems when they run an old version.
Another issue was that the developers have provided a public database, and without explicit instructions on how to delete the test entities or how to provision another database to connect to.
Therefore, in order to execute our tests faster and have a clean state on each run, we have contacted the developers for the database dump for the production database so that we can provision a similar database.

\subsection{Acceptance Test Setup}
We have used Cypress to create the acceptance tests for the project, which is a library that allows us to define browser-based tests rapidly.
We have used Typescript while creating our tests for typing validation in Javascript.
We have created a custom command in Cypress to allow us easily create users with certain properties.
For realistically random addresses we have pulled the street name list for all the streets in Milan from OpenStreetMaps API and saved it in Cypress as a fixture.
We have created schema instructions to reset the database each time it is connected to, and provided instructions in acceptanceTest.md on how to create a database and run the tests.
We have used an extra QR code reader library to read the content of QR codes generated by the server, without having to interfere with the server's code.
