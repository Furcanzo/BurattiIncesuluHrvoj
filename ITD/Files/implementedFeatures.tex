This section explains all implemented requirements of the CLup application alongside with motivations for including these specific featurs.

\subsection{Create Store}

Create store function is used by the Back Office User to create the first instance of a new store that joins the CLup application.
It is a part of the BackOfficeService class which is used for generating new stores and for generating employee with the store.
The store cannot exist without the employee in charge of it.
The store is generated with the related pieces of information such as its name, description,
location in a form of longitude and latitude, number of maximal customers and with a timeout for the line number tickets.
Methods that are used to create and generate a store with an employee are createStore, generateStore, and generateEmployeeWithStore.
The creation of the store is a vital function of the CLup application since the main purpose of the application is to create a line spot reservation system.

\subsection{Update Store Information}
Update store information is a function used by the manager to update the store information.
Through the editing form, a manager can update general information about the store, with the current values already pre-filled.
Besides all the information's that a store has since it was generated in the first place, it is now possible to add partner stores and working hours.
Update store is also a vital function for the CLup application to work as intended.
A store needs to have working hours to be able to generate line number tickets.

\subsection{Book Future Line Number}
Book future line number is used by the customer to book their visit to the store.
It belongs to the CustomerService class and uses the method called generateLineNumber to generate new line numbers.
To simplify the process of generating new line numbers, product categories have been omitted.
Product categories are nice to have feature but aren’t vital for the CLup application to work.
Book future line number is an important function of the CLup application.
Without it, the customer couldn’t book a visit to the desired store.

\subsection{Retrieve Line Number}
Retrieve line number is a function used by the clerk or by the customer to retrieve the first available line number from the system.
It’s a part of a CustomerService class and it is somewhat similar to the book future line number function.
Retrieve line number is used by the clerk when the customer approaches the store without a valid line number ticket and asks for the ticket to be generated by the clerk.
Customers may as well use the retrieve line number function if they want to reserve the first available spot at chosen store.
Retrieve line number is an important function of the CLup application as well as book future line number. They are both the core features of the CLup application in order to

\subsection{Add Staff Member}
Add staff member is used by the manager to generate new staff members. The manager needs to fill the relevant information such as email, storeID and the role of the new staff member.
The new role of a member can either be a manager or a clerk. New staff member is created with the generateEmployee method.
Generation of the new employee is important for the CLup application, but not vital.
A store can function with only having one manager created by the back office user if it doesn’t need clerks for scanning and creating line number tickets at the entrance of the store.

\subsection{Monitor Customers}
Monitor Customers is a simple but yet effective function which can be used by the manager to simply retrieve the number of customers in the store for the moment.
It uses monitorLive method from the EmployeeService class. It is not a vital function for the CLup application to work but it was easy to implement and can be handy for the manager.
Since all the customers must scan their line number ticket when entering and existing store, number of customers in a moment is easily calculated.

\subsection{Grant Access}
Grant Access is used for scanning customers QR codes while entering and exiting the store. It has two methods defined in the EmployeeServices class.
Checkin is used when the customer wants to enter the store. Customers line number status is then changed from “WAITING” to “VISITING”. When the customer exits the store,
checkout method is used and customers line number status is changed from “VISITING” to “VISITED”.
Grant Access is a vital function for CLup application because it ensures that the store isn’t overcrowded.
