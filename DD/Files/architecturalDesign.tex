\subsection{Overview}
% High-level components and their interaction
% TODO: @Hrvoje
\subsection{Component view}
In the following section, components used to implement different functionalities of the system is described, aided with component diagrams demonstrating their separation and interactions within each other and with other external interfaces.
\begin{figure}[H]
    \centering
    \includegraphics[height=0.4\textwidth]{Images/ComponentDiagrams/Overall.png}
    \caption{Component Diagram for the overall system}
    \label{fig:CDOverall}
\end{figure}
\nameref{fig:CDOverall} provides an overall view to the components present in the system and the connections between these components to correctly realize the decisions provided in this document.
The external integrations of the system, with the general communication interfaces they communicate with other components are provided, however the details for the Application Server will be presented following, only, considering that the main application logic is executed through the components residing in it.
The client is a thin-client built to interact with and display information directly sent through the endpoints.
Therefore, it is provided as one component that is unnecessary to split into sub-components.
\begin{figure}[H]
    \centering
    \includegraphics[height=0.4\textwidth]{Images/ComponentDiagrams/ApplicationServer.png}
    \caption{Component Diagram for the Application Server}
    \label{fig:CDApplicationServer}
\end{figure}
\nameref{fig:CDApplicationServer} provides an overall view for the components inside the Application Server.
The components separate the functionality into three domains, namely Line Numbers, Users and Stores.
All other entities that exist in the system are included into one of these domains based on their relevance.
All the user facing functionalities of the system are exposed through the REST endpoint, which uses a router interface to route each request to the domain component it belongs to.
All domain components use JPA to persist their domain data structures, and components that require to send e-mails use the SMTP endpoint to do so.
The OccupancyForecaster is a component that features only one function: periodically reading the database for entry and exit records of customers and updating the store accordingly.
Therefore, it is not split further into components.
\begin{figure}[H]
    \centering
    \includegraphics[height=0.4\textwidth]{Images/ComponentDiagrams/LineNumberManager.png}
    \caption{Component Diagram for the LineNumberManager}
    \label{fig:CDLineNumberManager}
\end{figure}
\nameref{fig:CDLineNumberManager} provides a detailed view over the domain component for line numbers.
It allows the management of any sort of query related to line numbers by clerks and customers.
This component is further divided into following sub-components to increase the granularity of actions performed on line numbers:
\begin{itemize}
    \item \textbf{PhysicalLineNumberGenerator}: This component exists to provide an interface for clerks to generate line numbers, it's interface handles the incoming request by generating and persisting a new ticket for the to-be-printed line number.
    \item \textbf{LineNumberScheduler}: This component exists to allow the users of the system to book a line number from their home.
    It realizes its functionality through connecting to the same interface of the persistence provider, however it is capable of handling more complex requests, including custom product or category selection and time slot allocation.
    Furthermore, it registers the ticket to the LineNumberInvalidator to be invalidated after the set timeout minutes have passed from the time slot.
    \item \textbf{LineNumberInvalidator}: This component acts as a helper component to LineNumberScheduler.
    It schedules the invalidation of the scheduled tickets, so that the invalidation can occur asynchronously.
    This component is created to separate this responsibility from an active user facing component, all which have the main responsibility to provide a synchronous response to all the users' needs.
    \item \textbf{LineNumberVerifier}: This component is used to handle the transactions regarding customer QR code verification conducted by the Clerk.
    It is used to register the entrance and exit of customers with their QR codes.
    \item \textbf{LineNumberFa\c{c}ade}: This component is used by the customers that want to query detailed information regarding the line numbers that they have.
    The requests handled by this endpoint is directly mapped to the client application's needs.

\end{itemize}
\begin{figure}[H]
    \centering
    \includegraphics[height=0.4\textwidth]{Images/ComponentDiagrams/StoreManager.png}
    \caption{Component Diagram for the StoreManager}
    \label{fig:CDStoreManager}
\end{figure}
\nameref{fig:CDStoreManager} provides a detailed view over the domain component for handling requests related to the store.
The wrapping component allows querying all relevant information about the store by all, and furthermore, harbors the logic for the manager users to update different aspects related to the store, such as basic information, time slots, configuration and system stop scheduling.
This component, apart from having mailing, routing and database interfaces, exposes an interface to the OccupancyProvider to retrieve updated information about the occupancy forecast. % TODO move to subcomponent
This component is divided into following components to decrease cohesion between tasks to be performed:
\begin{itemize}
    \item \textbf{StoreDetailsRepository}: This component is responsible for carrying out any direct query to the store data and all of the included classes, which are products, categories and time slots.
    There are two exported interfaces available to be used via HTTP.
    The query interface allows all the system users to view relevant information for the store, while the manager has access to the other interface allowing them to update the information.
    The exposed interface to the OccupancyProvider allows the OccupancyForecaster to store updated information regarding the future predictions for the store.
    \item \textbf{SystemStopCoordinator}: This component is responsible for carrying out all the actions that are necessary to perform or schedule a system stop, that are removing the specific time slot and sending e-mails to customers who has already booked those time slots.
\end{itemize}

\begin{figure}[H]
\centering
\includegraphics[height=0.4\textwidth]{Images/ComponentDiagrams/UserManager.png}
\caption{Component Diagram for the UserManager}
\label{fig:CDUserManager}
\end{figure}
\nameref{fig:CDUserManager} provides a detailed view of the sub-components related to requests that can be performed on users.
This component not only acts as a user registry, but also provides the authentication interfaces required, since its logic encompasses handling user data and authentication is done via user's email address, which relates these functions.
To seperate responsibility on different actions to be performed on the system, this component is divided into specific sub-components, that are:
\begin{itemize}
    \item \textbf{AuthenticationGateway}: This component acts as a medium to authenticate the user and generate the necessary authorization tokens for future use.
    \item \textbf{StaffRepository}: This component handles all the requests related to adding and removing staff members to stores.
    Since, in our design, different flavors of users are implemented using the same basis, the creation and removal of manager and clerks are similar from the architectural point of view.
    Furthermore, since the implementation of user creation is similar to that of the customers, a common factory component is provided to prevent duplication.
    \item \textbf{CustomerRegistrar}: This component handles the requests specific to registering new customers into the system.
    The component is responsible for calling the shared UserFactory to create a customer user and send the customer a welcome email.
    \item \textbf{UserFactory}: This component acts as a common interface and as a factory to create users.
    It is used by the components mentioned above to introduce new users to the system.
\end{itemize}
% Component diagram combining all components and each component having it's subcomponents in a different diagram with explanations
\subsection{Deployment view}
% TODO: @Roberto
% Deployment diagram with explaining each tier
\subsection{Runtime view}
% You can use sequence diagrams to describe the way components interact to accomplish specific tasks typically related to your use cases

% Main use cases with sequence diagrams indicating relations of each diagram to one another.

\subsection{Component interfaces}

% Class diagram with only methods demonstrating how components interact with each other.
% Also an ER or Class diagram to indicate data.
\subsection{Selected architectural styles and patterns}
% Please explain which styles/patterns you used, why, and how
\subsection{Other design decisions}

