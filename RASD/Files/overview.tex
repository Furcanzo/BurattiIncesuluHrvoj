\subsection{Product Perspective}
% Introductory text describing system's integration with other products, considering shared phenomena.

% Text for Class Diagram
% Class diagram

% TODO: @Ozan If have time start this.

% For each core feature (which are they? all?):
%   - Scenario
%   - State Chart
%   - Explanation

% TODO: @Ozan go with these.
% Core features: Future book line number, now retrieve line number, Guest arrives to store, System stop (Shop's on fire), schedule system stop (don't forget to notify customer),

% here  we  include scenarios  and further details on the shared phenomena and a domain model (class diagrams and state charts)


\subsection{Product functions}
% TODO: @Hrvoje do these 3

% subsubsections with functions of (some?) requirements
% here we include the most important requirements
% for requirements use the R_1, R_2, R_3 syntax

%As stated in the IEEE specification
%Provide a summary of the major functions that the software will perform.
%For example, an SRS for an accounting program may use this part to address customer account maintenance, customer statement, and invoice preparation without mentioning the vast amount of detail that each of those functions requires.
%Sometimes the function summary that is necessary for this part can be taken directly from the section of the higher-level specification (if one exists) that allocates particular functions to the software product.
%The product functions should be organized in a way that makes the list of functions understandable to the acquirer or to anyone else reading the document for the first time.

This section describes some basic functions of the software that are the most important ones and which sum up the whole idea of the CLup application.

\subsubsection {Schedule store visit}
Since the start of the Covid-19 pandemic, keeping physical distance from unknown people has never been more important. A lot of new restrictions were introduced, which were mainly oriented towards keeping people physicly separated. Stores now have a restriction on a number of people they could let in. To prevent people from waiting in lines on the streets CLup application comes in handy. It allows customers to schedule a visit to the store in their desired time slot if available. Each customer can registrate through CLup application and then reserve his visit to the store. Furthermore a customer can choose which products or products section he is going to buy and for how long his shopping will approximatly last. After a customer has successfully scheduled his visit time he will recieve a line number which he is obligated to show in order to enter the store.

\subsubsection {Line number ticket entry}
Before entering the store, a customer is required to show his line number ticket to a clerk who will then proceede to scan his ticket. If the customer has arrived in the scheduled time slot, he will be able to enter the store. If the customer does not have a line number ticket, clerk will provide him with one but he will not be able to enter the store immediately unless the time slot for that moment is not occupied with other customers. To sum up, a customer with scheduled store visit is able to visit the store in scheduled time without the need of waiting in line in front of a store while customer without the line number ticket depends on the free available time slots.

\subsubsection {Store managment}
Through the CLup application, manager is able to edit all the store information such as location specific information, maximum number of customers in the store at any given time, opening and closing hours of the store per each day, line number timeout and much more. Customers can see working hours and in store occupancy and therefore they are able to visit the store in desired time if available. Manager can also select which product categories or products are available at the store and by doing so customers can know precisely in which store to go based on the products available.

\subsection{User characteristics}
% User roles: Manager & User & Clerk (& maybe unregistered user?)
% People need to be able to use Tickets, maybe: %90 / %10

%Describe those general characteristics of the intended groups of users of the product including characteristics that
%may influence usability, such as educational level, experience, disabilities, and technical expertise.
%This description should not state specific requirements, but rather should state the reasons why certain specific requirements
%are later specified in specific requirements

\subsubsection {Customer}
A person who is registered on the CLup application and is able to use its functionalities to be able to successfully schedule a visit to the store in desired visit time if available. The amount of product training needed for a customer is none since the level of technical expertise and educational background is unknown as the range of customers include all demographics. The only skill needed by a user is the ability to use the application.

\subsubsection {Clerk}
An employee of the store with basic knowledge about the application. Clerk must be able to use its functionalities so he could handle all physical visits of customers.

\subsubsection {Manager}
A special employee of the store with basic knowledge about the application. Manager is in charge of managing a particular store through CLup application, he doesn't need to be a real store manager as well.

\subsection{Assumptions,dependencies and constraints}

%List each of the factors that affect the requirements stated in the SRS.
%These factors are not design constraints on the software but any changes to these factors can affect the requirements in the SRS.
%For example, an assumption may be that a specific operating system will be available on the hardware designated for the software product.
%If, in fact, the operating system is not available, the SRS would then have to change accordingly.

\subsubsection{Domain Assumptions}

\begin{itemize}
    \item \textbf{$D_1$} \%80 of the customers and all clerks and managers have basic ICT skills, has an email address that they are willing to use to authenticate to the system and has a smart phone or equivalent device that can connect to the Internet, have a browser that supports UTF-8, display QR codes and has a mapping application. % All cases
    \item \textbf{$D_2$} Locations will not be visited by no more than 1000 people in any time slot. % Line number limit, we can provide 000-999 as numbers
    \item \textbf{$D_3$} \%98 of the customers will arrive at the given location either without a ticket or with a ticket that has not timed out. % Timeout policy
    \item \textbf{$D_4$} E-mail addresses are not shared by multiple users of the system. % Registeration, user info
    \item \textbf{$D_5$} Clerks' mobile devices are equipped with at least one camera that the system can use. % For QR code detection
    \item \textbf{$D_6$} All users have a basic understanding of how the line numbering system works and respects the ordering provided by the system. % The core use case
    \item \textbf{$D_7$} Managers have an estimate for the amount of reservations that their location can at most have. % To set a maximum amount of clients
    \item \textbf{$D_8$} Managers' device has location services that has a location acquisition error for no more than 20 meters. % To set the location accurately
    \item \textbf{$D_9$} Clerks are constantly monitoring the locations entrances and exits. % For QR code reading to take customers in and out
    \item \textbf{$D_{10}$} Locations have printing equipments that are in 5 meters range of all the entrances that can print QR codes and line numbers. % To print tickets
    \item \textbf{$D_{12}$} At least one manager is available in the location during the working hours. % To monitor and shutdown during emergencies
    \item \textbf{$D_{13}$} The customer's entry and exit to the store is determined by whether the clerks have checked them in and out.
    \item \textbf{$D_{14}$} The customer has their line number or line number ticket available with them through their visit, including their exit from the store % (no dead battery, printed ticket loss, etc)
\end{itemize}
% here we include domain assumptions
% use the D_1, D_2,... syntax
\subsubsection{Dependencies}
% TODO: @Hrvoje these 2 too.
% Maps API
% What external libraries, tools, integrations does the system rely on

Application will be running on a server provided by the hosting provider. Application will be dependent on Map API which will be chosen later in design document.

\subsubsection{Constraints}
% Maps API failure
% All user and location info can be represented with UTF-8 character encoding.
% What are the limits imposed by the environment, regulatory policies, hardware & software limitations, etc...

Application interface should be user friendly and intuitive enough for all demographics to use, it should also be mobile friendly as well as desktop friendly. Application should work on smartphones, tablets and desktop devices in order to be available to as many devices as possible. Application should be developed and fully functional before the end of Covid-19 crises.
