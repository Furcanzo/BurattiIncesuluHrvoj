\subsection{Purpose}
% here we include the goals of the project

\subsubsection{Description of the Proposed System}
The project "CLup - Customer Line up" is a line spot reservation system planned to be used by managers, clerks, and customers of many local vendors and chains.
The system aims to come up with a handy solution for the ongoing issue of proper social distancing management, particularly in grocery shopping, by providing assistance to cope with the customer load for managers and helping customers access products in a safe and controlled manner.

In particular, users will be able to see the locations, get a line number, and book in advance for the grocery stores they would like to visit.
Once assigned a line number, the customer will track the estimated time of arrival of the line and wait for the notification that informs about his or her line's forthcoming arrival; hence waiting time in the line in the crowd is minimum.
Also, "CLup" provides uniquely generated QR codes per the line number, which can be utilized by the store managers as a proper monitoring tool in the entrances and exits of the locations.
The product's general purpose is to keep the congestion levels in line with the locations at a minimum via providing useful features for all the users.




\subsubsection{Goals}

\begin{itemize}
    \item \textbf{$G_{1}$} Customers can issue a line number for a location.
    \item \textbf{$G_{2}$} Customers can issue line numbers for their future visits.
    \item \textbf{$G_{3}$} Customers can detail their visits by category or product.
    \item \textbf{$G_{4}$} Customers can plan their visit to the store.
    \item \textbf{$G_{5}$} Customers may prefer to use alternative time slots or partner stores for their visit.
    \item \textbf{$G_{6}$} Managers can prevent customers from issuing line numbers.
    \item \textbf{$G_{7}$} Managers can customize the system to allow optimizations for increased granularity, flow control, and time slot forecasting.
    \item \textbf{$G_{8}$} Clerks and Managers can monitor the customers through their entrance and exits.
    \item \textbf{$G_{9}$} Customers can obtain printed line number tickets.
\end{itemize}

% Manager:
% Set the estimated customer number to be in the location (based on govt. regulations, space availability etc.)
% Set the opening and closing hours of the store per day.
% Set the in-shop location & categories of products in the location
% Set the limit of reservation per customer per month, per day, per week
% Set the location of the store
% Add other chain members or partner stores
% Observe the current amount of customers in the location.
% Stop the system from issuing more tickets for a given day in case of an overflow or emergency
% Schedule a system stop.
% Set the timeout

% Clerk:
% Monitor the entrance and exit of customers with line numbers, by scanning their QR codes.
% Print line numbers with QR code on them for the physically arriving customers who doesn't have a virtual ticket.

% Customer:
% Retrieve a direct line number for a location
% Generate a QR code for entry
% See an ETA, like 30 mins are left for your line number
% View the location of the store to plan their trip
% Book a future line number for a day
% See the occupancy of the location for different time and day.
% View the alternative slot suggestions for the visit planned
% View the alternative slots available in near chain or partner stores
% Set the categories, products of interest before visit for finer granularity.
% Set the estimated time for visit.
% See the amount of people currently in the store
% Set user details, name, surname
% View forecasted occupancy for time slots.

% System:
% Forecast the occupancy of store per time slot
% Timeouts the line numbers when not registered for given timeout by manager.

% 80/20 percent split between bookings and physical appearence on domain assumptions for us to schedule in an easy way.
\subsection{Scope}

\subsubsection{Targeted Users}

"CLup" aims to resolve the problem of Customers queueing up in front of a location, without control of availability of place in the location and future contact tracing by managers.
\\[0.5cm]
\textbf{Customer:} \\

Customers will be able to obtain specific line numbers for various locations using CLup, which they can track the estimated time available with and also view the location on a map application to plan their visit.
Customers can further obtain line numbers for future visits, based on the system's info about the availability of free spots on the specific time intervals.
They may prefer to visit a different branch of the same chain.
Furthermore, they can provide specific products they intend to purchase or set an estimated time for their visit to allow finer granularity.
Some customers may also prefer to obtain a line number upon visiting the store physically.
To plan their visit in a time slot where the location will be less crowded, the users can see the location's occupancy based on already taken line numbers and forecasts provided by the system.
\\[0.5cm]
\textbf{Clerk:} \\
The clerk (which can be a shopping assistant or a security detail) can monitor customers' flow and manually intervene in case of missing line numbers via performing manual checkout for a specific customer or by printing line numbers physically for some customers.
\\[0.5cm]
\textbf{Manager:} \\

The location manager (which could be an actual manager or someone responsible for handling customer management) can provide details regarding the availability of products and the location in general, by setting the opening hours, maximum allowed customers in the shop, in-shop location of different products and categories, the maximum amount of reservations that can be made per customer, line number timeout and the location.
Also, for chains and relevant stores, the manager can add chain members for the location.

\subsubsection{Relevant Phenomena}

\begin{table}[H]
    \begin{tabular}{|p{4cm}|p{4cm}|p{4cm}|}
        \hline
        \textbf{Phenomenon}    & \textbf{World / Machine}       & \textbf{Shared } \\ \hline
        Line Number & Machine & Yes \\ \hline
        Line Number Ticket & World & No \\ \hline
        Product & World & Yes \\ \hline
        Product Category & World & Yes \\ \hline
        In-store Location & World & Yes \\ \hline
        Occupancy Forecast & Machine & No \\ \hline
        Store & World & Yes \\ \hline
        Time Slot & Machine & Yes \\ \hline
        Ticket Printer & World & No \\ \hline
        Line Number Timer & Machine & No \\ \hline
        Customer Scheduling Algorithm & Machine & No \\ \hline

        % Line template:
        %          &               &  \\ \hline
    \end{tabular}
    \caption{Relevant Phenomena of CLup}
\end{table}

% here we include an analysis of the world and of the shared phenomena


\subsection{Definitions, Acronyms, Abbreviations}
\subsubsection{Definitions}
\begin{itemize}
    \item \textit{Location}: the physical location of the business that operates the line reservation system
    \item \textit{Manager}: the user in charge of executive action within the location
    \item \textit{Customer}: the user to visit the location
    \item \textit{Clerk}: the user in charge of handling the entrance and exit of customers
    \item \textit{Visit Time}: the time interval in which a customer performs a visit to the location
    \item \textit{Line Number}: A number that indicates the ordering of a specific customer in the line
    \item \textit{Line Number Ticket}: A physical ticket printed that features the line number and the QR code.
    \item \textit{Time Slot}: Specific intervals of time determined by the opening hours and average visit time per customer.
    \item \textit{Partner Store}: A different location included in the same beneficiary chain of command (such as another member of the franchise or store chain) or a mutual agreement with the specific location
    \item \textit{Product}: Any item, items, service or services demanded by the customer, and provided by the store to the customer.
    \item \textit{In-store Location}: A location of a specific product or a product category inside the store.
    \item \textit{Working hours}: The time intervals that the store is open during each day.
    \item \textit{Maps API}: A third-party mapping service implementation used for location tracking
\end{itemize}
\subsubsection{Acronyms}
\begin{itemize}
    \item \textbf{RASD}: Requirement Analysis and Specification Document
    \item \textbf{QR Code}: Quick Response Code
    \item \textbf{API}: Application Programming Interface
    \item \textit{ETA}: Estimated Time of Arrival
\end{itemize}
\subsubsection{Abbreviations}
\begin{itemize}
    \item \textbf{$G_n$}: $n^{th}$ goal
    \item \textbf{$D_n$}: $n^{th}$ domain assumption
    \item \textbf{$R_n$}: $n^{th}$ functional requirement
\end{itemize}
\subsection{Revision history}


\subsection{Reference Documents}
\begin{itemize}
    \item \href{https://beep.metid.polimi.it/documents/121843524/23d1869d-ab17-4e36-979e-f1ccbc59be24}{Specification Document: R\&DD Assignment AY 2020-2021}
    \item \href{https://standards.ieee.org/standard/29148-2011.html}{IEEE Std 830-1998: IEEE Recommended Practice for Software Requirements Specifications}
    \item \href{https://www.iso.org/standard/62021.html}{ISO/IEC 18004:2015 QR Code bar code symbology specification}
    % \item \href{https://developers.google.com/maps/documentation/javascript/overview}{Maps API documentation}
    \item \href{https://www.iso.org/iso-8601-date-and-time-format.html}{IEEE Std 8601 Date and time format}
\end{itemize}


\subsection{Document Structure}
This document is composed of six sections, each with the purpose described below: \\
\begin{itemize}
    \item \textbf{Introduction}: This section provides an introduction of the problem, the scope of the project with details regarding the goals, target users, and phenomena.
    The project's goals are formulated per the description of actions and actors in the Specification Document.
    Within the project's scope, the properties and duties of different users and user groups are described in the Target Users section.
    Furthermore, under the project's scope, the relevant phenomena of the project are presented through their relevance to the world and the machine.
    \item \textbf{Overall Description}: This section builds upon the introduction section by providing an expansion of the system's scope and relevant functionalities.
    The fundamental perspective is provided considering the shared phenomena between the world and the system and the system's integrations with other parties and products.
    The overall actions that can be conducted over the system are described using a detailed class diagram, portraying all required components of the system, their inner state, and provided functionalities for other components to facilitate their purpose.
    All core features of the system are listed and explained using user scenarios to understand better the real-world condition on which the use case should apply, a state chart for demonstrating the relationship between various states the system can be in during the execution of the provided core feature scenario.
    Next, the system's requirements to archive the goals mentioned above and allow the provided scenarios' execution are given.
    Based on their roles in the system, the user characteristics are further evaluated in this section through the means of the needs that they have, how the system archives those needs via its requirements and goals.
    No system can be designed without a broad idea of what domain it exists and how much the environment limits its ability to perform its features.
    Ergo, we provide the standard assumptions of the domain, the system's dependencies to external components, and the world's limits on the system.
    \item \textbf{Specific Requirements}: This section is the main section of the document.
    The external interfaces that the system requires to function correctly are presented with details and additional mockups if needed.
    The interfaces section is mainly concentrated on different views of the user interfaces as the user-facing part is the main way to integrate with the system.
    In this section, the system's functional requirements are presented in detail, a common use case diagram is provided, and each use-case of the system is analyzed with details on a use case description table and a sequence diagram.
    The performance requirements of the system, with a focus on development speed and scalability, are provided.
    The system's design constraints, with a focus on different standard compliances, hardware limitations of the system to function, and other constraints, like GDPR, are also elaborated in this section.
    The vertical aspects that need consideration are evaluated under the Software System Attributes subsection of this document, with an emphasis on Reliability, Availability, Security, Maintainability, and Portability.

    \item \textbf{Formal Analysis Using Alloy}: This section features various models built, and hypothesis verified using Alloy.
    This section demonstrates that some aspects of the requirements document can be formally proven correct and provide additional information about the proposed system to the engineers.
    \item \textbf{Effort Spent}: This section features the effort table, in which all team members provide a rough estimation of the time spent on the creation of the various sections of the document.
    \item \textbf{References}: This section features different reference materials referred to in this document.
\end{itemize}
